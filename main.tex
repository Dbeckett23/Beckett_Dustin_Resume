\documentclass[]{cv-style}          
\usepackage{pifont}
\usepackage[absolute,overlay,showboxes]{textpos}
\usepackage[document]{ragged2e}
\usepackage{graphicx}
\usepackage{fontawesome}
\usepackage{hyperref}

\sethyphenation[variant=british]{english}{}

\begin{document}
\header{Dustin}{Beckett}

%----------------------------------------------------------------------------------------
%	SIDEBAR SECTION  -- In the aside, each new line forces a line break
%----------------------------------------------------------------------------------------

\begin{aside}
%
\begin{tabular*}{\textwidth}{@{}l@{\extracolsep{\fill}}r@{}}

\section{Contact}

\hspace{1mm}Cell:
\hspace{1mm}+1 (734) 864-2001 
~
\hspace{1mm}E-mail:
\hspace{1mm}DustinWBeckett@gmail.com
% ~
% \hspace{1mm}\href{www.linkedin.com/in/dustin-beckett}{LinkedIn}\hspace{1mm}\faLinkedinSquare
% ~
% \hspace{1mm}Website:
% \hspace{1mm}www.dustin-beckett.com
~
\section{Programming Languages}
\hspace{1mm}\textbf{Java}
\hspace{1mm}Python
%
~
\section{Skills}
\hspace{1mm}Distributed Systems
\hspace{1mm}Test Driven Development
% \hspace{1mm}Object-Oriented Prog.
\hspace{1mm}Agile Development
\hspace{1mm}Algorithms
\hspace{1mm}Data Structures
%
~
\section{Tools}
\hspace{1mm}Git
\hspace{1mm}Maven
\hspace{1mm}Jenkins
% \hspace{1mm}Make
\hspace{1mm}AWS
\hspace{1mm}New Relic
\hspace{1mm}Sumo Logic
\hspace{1mm}IntelliJ IDEA
~
\section{Computer Skills}
\hspace{1mm}Linux
\hspace{1mm}Windows
\hspace{1mm}MacOS
~
\section{Soft Skills}
\hspace{1mm}Communication
\hspace{1mm}Problem solving
\hspace{1mm}Team player
\hspace{1mm}Resilient
\hspace{1mm}Willingness to Learn
\hspace{1mm}Dependability
~
\section{Interests}
\hspace{1mm}Rust
\hspace{1mm}WebAssembly
\hspace{1mm}System Design
\hspace{1mm}Vim

\end{tabular*}
\end{aside}

%----------------------------------------------------------------------------------------
%	WORK EXPERIENCE SECTION
%----------------------------------------------------------------------------------------
\section{Experience}
\begin{entrylist}
%------------------------------------------------
\entry
{2020--Now}
{Genesys}
{Remote (Michigan)}
{Software Engineer\\
Worked on the Genesys Cloud Public API team.
\begin{itemize}
% Migrated session and image APIs
    \item Revamped cache logic for a high-profile client, resulting in a significant reduction in the overall load on our backend service and an improvement in the user experience
    % \item Revamped our clients caching strategy by adding automatic refresh and fallback logic
    \item Implemented an AWS Lambda function to handle automatic image scaling and storage in S3, resulting in improved image delivery times and reduced infrastructure costs
    \item Successfully remedied production PagerDuty alerts while on call by utilizing tools such as New Relic and Sumo Logic to quickly identify the root cause and take appropriate actions, minimizing downtime and ensuring high system availability for critical services
    % \item Overhauled our platforms image APIs to be more efficient and user friendly by reducing the number of API calls and improving the response body
    \item Led the overhaul of our platform's image APIs to optimize efficiency and user experience by cutting the number of API calls in half and improving the response body
    \item Reviewed and provided feedback on hundreds of pull requests as a primary reviewer for a public API repository, ensuring adherence to best practices and maintaining high code quality standards
\end{itemize}}
%------------------------------------------------
\entry
{2019}
{Genesys}
{Indianapolis, Indiana}
{Software Engineer, Intern\\
Worked on the Genesys Cloud platform Public API team.
\begin{itemize}
    % \item Improved the Public API's property management system by creating a class that handled retrieving properties from local YAML files and a Amazon DynamoDB instance
    \item Improved our Public API's property management system by creating a new class that streamlined property retrieval from local YAML files and DynamoDB, resulting in increased stability for the platform
    \item Explored new ways to automate version bumping using Jenkins and Grunt when making commits to a project's master branch
    % \item Developed my skills in a DevOps work environment by working on tickets and automating tasks such as building, testing, and deploying software
    \item Advanced proficiency in a DevOps work environment through the completion of tickets and automation of tasks such as building, testing, and deploying software, resulting in more and higher quality software releases
    
    % \item Built additions for software that had a codebase of over half a million lines of code
    \item Accelerated the development process by using development tools such as Maven, Git, IntelliJ IDEA, and Jenkins
\end{itemize}}
%------------------------------------------------
% \entry
% {2018}
% {Undergraduate Research}
% {University of Michigan - Ann Arbor}
% {Research Assistant\\
% Installed and modified software on a server at the University of Michigan to be used by archaeologists in Italy. 
% \begin{itemize}
%     \item Satisfied design specifications by modifying existing archaeological software to match the requirements needed by archaeologists
%     \item Connected archaeological data by launching a website that allowed archaeologists to enter data at the dig site
%     \item Managed the SQL database that stored the data entered by archaeologists
% \end{itemize}}
%------------------------------------------------
\end{entrylist}


%----------------------------------------------------------------------------------------
%	PROJECT EXPERIENCE SECTION
%----------------------------------------------------------------------------------------
\section{Projects}
\begin{entrylist}

% %------------------------------------------------
\entry
{2023}
{Password Manager}
{Rust}
{Developed a program to securely store password using rust\\
\begin{itemize}
    \item Developed a Rust-based password manager using the argon2 hashing algorithm and a SQLite database for secure password storage and retrieval
    \item Utilized generics and zero-sized types to ensure the password manager's APIs could not be misused, increasing the overall security of the application
    \item Implemented best practices for password security, including salting and iterating the argon2 algorithm, providing industry-standard security to users
    \item Conducted rigorous testing and code review to ensure the password manager was stable and secure, resulting in a reliable and user-friendly application
\end{itemize}}

% %------------------------------------------------
% \entry
% {2018}
% {SillyQL}
% {EECS 281}
% {Developed a program that acts as a relational database that can store large amounts of data and can be accessed quickly based on a given query.
% \begin{itemize}
%     \item Expanded my knowledge of how to solve different types of problems by utilizing the C++ Standard Template Library (STL)
%     \item Ensured rapid data access by using the C++ STL unordered map to search for data elements in the database
%     \item Utilized the speed of hashing to create a database that avoids a linear search to query a table or a set of tables in the database
% \end{itemize}}
% %------------------------------------------------
% \entry
% {2018}
% {Machine Learning Program}
% {EECS 280}
% {Developed a program that read in data from Piazza, a Q\&A web service, that used a version of Bayes' theorem, to predict which project label each question was asked in.
% \begin{itemize}
%     \item Predicted forum labels with a 75\% accuracy using the supervised learning method of machine learning
%     \item Improved my understanding of data structures, and how to apply them to solve different types of problems
%     \item Enhanced my ability to extract large amounts of data from files, and how to apply the extracted data to my program
% \end{itemize}}

% %------------------------------------------------
\end{entrylist}
%----------------------------------------------------------------------------------------
%	EDUCATION SECTION
%----------------------------------------------------------------------------------------

\section{Education}

\begin{entrylist}
%------------------------------------------------
\entry
{2017--2020}
{B.Sc. {\normalfont in Computer Science}}
{University of Michigan - Ann Arbor}
{\vspace{-0.3cm}}
%------------------------------------------------
\entry
{2015--2017}
{A.Sc. {\normalfont in Computer Science - Programming in Java}}
{Washtenaw Community College}

%------------------------------------------------
\end{entrylist}

%--------------------------------------------------------------------------------------
\end{document}
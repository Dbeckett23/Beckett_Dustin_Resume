\documentclass[]{cv-style}          
\usepackage{pifont}
\usepackage[absolute,overlay,showboxes]{textpos}
\usepackage[document]{ragged2e}
\usepackage{graphicx}
\usepackage{fontawesome}
\usepackage{hyperref}

\sethyphenation[variant=british]{english}{}

\begin{document}
\header{Dustin}{Beckett}

%----------------------------------------------------------------------------------------
%	SIDEBAR SECTION  -- In the aside, each new line forces a line break
%----------------------------------------------------------------------------------------

\begin{aside}
%
\begin{tabular*}{\textwidth}{@{}l@{\extracolsep{\fill}}r@{}}

\section{Contact}

\hspace{1mm}Cell:
\hspace{1mm}+1 (734) 864-2001 
~
\hspace{1mm}E-mail:
\hspace{1mm}DustinWBeckett@gmail.com
% ~
% \hspace{1mm}\href{www.linkedin.com/in/dustin-beckett}{LinkedIn}\hspace{1mm}\faLinkedinSquare


% ~
% \hspace{1mm}Website:
% \hspace{1mm}www.dustin-beckett.com
~
\section{Programming Languages}
\hspace{1mm}\textbf{Java}
\hspace{1mm}Python
%
~
\section{Skills}
\hspace{1mm}Distributed Systems
\hspace{1mm}Test Driven Development
% \hspace{1mm}Object-Oriented Prog.
\hspace{1mm}Agile Development
\hspace{1mm}Algorithms
\hspace{1mm}Data Structures
%
~
\section{Tools}
\hspace{1mm}Git
\hspace{1mm}Maven
\hspace{1mm}Jenkins
% \hspace{1mm}Make
\hspace{1mm}AWS
\hspace{1mm}New Relic
\hspace{1mm}Sumo Logic
~
\section{Computer Skills}
\hspace{1mm}IntelliJ IDEA
\hspace{1mm}Linux
\hspace{1mm}Windows OS
\hspace{1mm}MacOS
~
\section{Business Skills}
\hspace{1mm}Communication
\hspace{1mm}Problem solving
\hspace{1mm}Team player
\hspace{1mm}Resilient
~
\section{Interests}
\hspace{1mm}Rust
\hspace{1mm}WebAssembly
\hspace{1mm}System Design
\hspace{1mm}Vim

\end{tabular*}
\end{aside}

%----------------------------------------------------------------------------------------
%	WORK EXPERIENCE SECTION
%----------------------------------------------------------------------------------------
\section{Experience}
\begin{entrylist}
%------------------------------------------------
\entry
{2020--Now}
{Genesys}
{Remote (Michigan)}
{Software Engineer\\
Worked on the Genesys Cloud Public API team.https://www.overleaf.com/project/5bfaeb95dd2484539c026ea8
\begin{itemize}
    \item Revamped our clients caching strategy by implementing automatic refresh and fallback logic
    \item Implemented an AWS Lambda to handle scaling images and storing them in S3 as part of an on going migration
    \item Remedied production PagerDuty alerts while on call using tools such as New Relic and Sumo Logic to determine the cause
    \item Overhauled our platforms image APIs to be more efficient and user friendly by reducting the number of API calls and improving the response
    \item Reviewed hundreds of pull requests as a primary reviewer for our public api repository
\end{itemize}}
%------------------------------------------------
\entry
{2019}
{Genesys}
{Indianapolis, Indiana}
{Software Engineer, Intern\\
Worked on the PureCloud plateform Public API team.
\begin{itemize}
    \item Improved the Public API's property management system by creating a class that handled retrieving properties from local YAML files and a Amazon DynamoDB instance
    \item Explored new ways to automate version bumping using Jenkins and Grunt when making commits to a project's master branch
    \item Developed my skills in a DevOps work environment by working on tickets and automating tasks such as building, testing, and deploying software
    \item Built additions for software that had a codebase of over half a million lines of code
    \item Accelerated the development process by using development tools such as Maven, Git, IntelliJ IDEA, and Jenkins
\end{itemize}}
%------------------------------------------------
% \entry
% {2018}
% {Undergraduate Research}
% {University of Michigan - Ann Arbor}
% {Research Assistant\\
% Installed and modified software on a server at the University of Michigan to be used by archaeologists in Italy. 
% \begin{itemize}
%     \item Satisfied design specifications by modifying existing archaeological software to match the requirements needed by archaeologists
%     \item Connected archaeological data by launching a website that allowed archaeologists to enter data at the dig site
%     \item Managed the SQL database that stored the data entered by archaeologists
% \end{itemize}}
%------------------------------------------------
\end{entrylist}


%----------------------------------------------------------------------------------------
%	PROJECT EXPERIENCE SECTION
%----------------------------------------------------------------------------------------
\section{Projects}
\begin{entrylist}

% %------------------------------------------------
\entry
{2023}
{Password Manager}
{Rust}
{Developed a program to securely store password using rust\\
\begin{itemize}
    \item 
    \item 
    \item 
\end{itemize}}

% %------------------------------------------------
% \entry
% {2018}
% {SillyQL}
% {EECS 281}
% {Developed a program that acts as a relational database that can store large amounts of data and can be accessed quickly based on a given query.
% \begin{itemize}
%     \item Expanded my knowledge of how to solve different types of problems by utilizing the C++ Standard Template Library (STL)
%     \item Ensured rapid data access by using the C++ STL unordered map to search for data elements in the database
%     \item Utilized the speed of hashing to create a database that avoids a linear search to query a table or a set of tables in the database
% \end{itemize}}
% %------------------------------------------------
% \entry
% {2018}
% {Machine Learning Program}
% {EECS 280}
% {Developed a program that read in data from Piazza, a Q\&A web service, that used a version of Bayes' theorem, to predict which project label each question was asked in.
% \begin{itemize}
%     \item Predicted forum labels with a 75\% accuracy using the supervised learning method of machine learning
%     \item Improved my understanding of data structures, and how to apply them to solve different types of problems
%     \item Enhanced my ability to extract large amounts of data from files, and how to apply the extracted data to my program
% \end{itemize}}

% %------------------------------------------------
\end{entrylist}
%----------------------------------------------------------------------------------------
%	EDUCATION SECTION
%----------------------------------------------------------------------------------------

\section{Education}

\begin{entrylist}
%------------------------------------------------
\entry
{2017--2020}
{B.Sc. {\normalfont in Computer Science}}
{University of Michigan - Ann Arbor}
{Relevant courses: Algorithms \& Data Structures, Databases}
{\vspace{-0.3cm}}
%------------------------------------------------
\entry
{2015--2017}
{A.Sc. {\normalfont in Computer Science - Programming in Java}}
{Washtenaw Community College}

%------------------------------------------------
\end{entrylist}

%--------------------------------------------------------------------------------------
\end{document}